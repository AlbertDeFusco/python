\documentclass[xcolor=table,10pt,final]{beamer}
\renewcommand\mathfamilydefault{\rmdefault}

\setbeamertemplate{navigation symbols}{}
\usepackage{amsmath,amsfonts,amssymb,pxfonts,xspace}
\usepackage{textcomp}
\usepackage{lmodern}
\usepackage{verbatim}
\usepackage{graphicx}
\usepackage{listings}
\usepackage[T1]{fontenc}

\lstset{
    basicstyle=\footnotesize,
    keywordstyle=\color[rgb]{0.1,0.8,0.1}\bfseries,
    commentstyle=\color{blue},
    numbers=left,
    stringstyle=\ttfamily\color{red!50!brown},
    showstringspaces=false}
\lstset{literate=%
   *{0}{{{\color{red!20!violet}0}}}1
    {1}{{{\color{red!20!violet}1}}}1
    {2}{{{\color{red!20!violet}2}}}1
    {3}{{{\color{red!20!violet}3}}}1
    {4}{{{\color{red!20!violet}4}}}1
    {5}{{{\color{red!20!violet}5}}}1
    {6}{{{\color{red!20!violet}6}}}1
    {7}{{{\color{red!20!violet}7}}}1
    {8}{{{\color{red!20!violet}8}}}1
    {9}{{{\color{red!20!violet}9}}}1
}



\begin{document}

\title{Python: beyond the basics}
\author{Albert DeFusco}
\date{\today}


\frame{\titlepage}

\begin{frame}[fragile]
  \frametitle{The Davidson Algorithm}
  \begin{equation*}
    \mathbf{Hc}^k = \lambda^k\mathbf{c}^k
  \end{equation*}
  \begin{enumerate}
    \item Guess an orthonormal subspace $\{\mathbf{b}_1,\mathbf{b}_2\}$
    \item Form and diagonalize the subspace matrix $\mathbf{G}$
      \begin{eqnarray*}
        \mathbf{G}_{ij} &=& \mathbf{b}_i^T\mathbf{Gb}_j\\
        \mathbf{G}\boldsymbol{\alpha}^k &=&  \rho^k\boldsymbol{\alpha}^k
      \end{eqnarray*}
    \item Compute the residue and correction vector for each state
      \begin{eqnarray*}
        \mathbf{r}^k &=& \sum_{i}\boldsymbol{\alpha}^k\left(\mathbf{Hb}_i - \rho^k\mathbf{b}_i\right)\\
        \boldsymbol{\delta}^k &=& -\left(\mathbf{D}-\rho^k\mathbf{I}\right)^{-1}\mathbf{r}^k
        \end{eqnarray*}
      \item Orthogonalize $\boldsymbol{\delta}^k$ to $\{\mathbf{b}\}$ and append to $\{\mathbf{b}\}$
      \item Goto 2 until $\rho^k$ is converged
        \begin{eqnarray*}
          \lambda^k &\approx& \rho^k\\
          \mathbf{c}^k &\approx& \sum_i\boldsymbol{\alpha}_i^k\mathbf{b}_i
        \end{eqnarray*}
  \end{enumerate}
\end{frame}

\begin{frame}[fragile]
  \frametitle{Classes}
  \begin{lstlisting}[language=python]
  class Molecule(object):
    """Molecules have a name and chemical formula"""
    #Print the above message with "print <instance>.__doc__"
    weight=0
    def __init__(self,name)
      self.name = name
    def setName(self,name):
      self.name = name
    def setFormula(self,formula):
      self.formula=formula
    def computeWeight(self)
      try:
        weight = f(formula)
      except:
        print "the chemical formula was not set"
        return None
   \end{lstlisting}
   \begin{itemize}
     \item Private members and methods are name mangled
       \begin{itemize}
         \item \lstinline[language=python]|__name| becomes \lstinline[language=python]|_classname_name| at runtime
         \item Somewhat taboo in Python culture
       \end{itemize}
   \end{itemize}
\end{frame}

\begin{frame}[fragile]
  \frametitle{Overloading}
  Change or the define the behavior of operations
  \begin{lstlisting}[language=python]
class Molecule(object):
...
  def __add__(self, other):
  return self.weight+other.weight,self.formula+other.formula
  \end{lstlisting}
\end{frame}

\begin{frame}[fragile]
  \frametitle{Inheritance}
  \begin{itemize}
    \item Child classes can be more {\it specific} than the parent\\
      \begin{lstlisting}[language=python]
class Chromophore(Molecule):
  """Chromophore is a special type of Molecule"""
  def absorption(self):
    """Compute absorption spectrum"""
    self.specturm = absorb(formula)
      \end{lstlisting}
    \item Subclasses can override the superclass$^\dagger$\\
  \begin{lstlisting}[language=python]
class Polymer(Molecule):
  """A polymer has a monomer (molecule) and a repeat"""
  def __init__(self,name,lenght):
    super(Polymer,self).__init__("poly"+name)
    self.lenght = lenght
  def grow(self,length):
    self.weight+=length*self.weight/self.length
    self.length+=length
  def setFormula(self,formula):
    self.formula="["+formula+"]"+str(self.length)
  def setWeight(self,weight):
    self.weight=weight*self.length
  \end{lstlisting}
  \end{itemize}
  $^\dagger${\scriptsize Polymorphism in Python is achieved when classes implement the same methods, which reduces the number of {\tt if} statements}
\end{frame}


\end{document}
