%%%%%%%%%%%
% April 12 notes
%
% what did I want to add to boolean?
% escapes in strings
% "where" in containers
% better list vs. tuple
% better mutable vs. immutable
%
% move arrays to just after pi
%  make examples and homework
%
% functions may stay
% classes can go to lecture 2
%



\documentclass[xcolor=table,10pt,final]{beamer}
\renewcommand\mathfamilydefault{\rmdefault}

\setbeamertemplate{navigation symbols}{}
\usepackage{amsmath,amsfonts,amssymb,pxfonts,xspace}
\usepackage{textcomp}
\usepackage{lmodern}
\usepackage{verbatim}
\usepackage{graphicx}
\usepackage{listings}
\usepackage[T1]{fontenc}

\lstset{
    basicstyle=\footnotesize,
    keywordstyle=\color[rgb]{0.1,0.8,0.1}\bfseries,
    commentstyle=\color{blue},
    numbers=left,
    stringstyle=\ttfamily\color{red!50!brown},
    showstringspaces=false}
\lstset{literate=%
   *{0}{{{\color{red!20!violet}0}}}1
    {1}{{{\color{red!20!violet}1}}}1
    {2}{{{\color{red!20!violet}2}}}1
    {3}{{{\color{red!20!violet}3}}}1
    {4}{{{\color{red!20!violet}4}}}1
    {5}{{{\color{red!20!violet}5}}}1
    {6}{{{\color{red!20!violet}6}}}1
    {7}{{{\color{red!20!violet}7}}}1
    {8}{{{\color{red!20!violet}8}}}1
    {9}{{{\color{red!20!violet}9}}}1
}



\begin{document}

\title{Python: beyond the basics}
\author{Albert DeFusco}
\date{\today}
\frame{\titlepage}

\begin{frame}
  \frametitle{What's so great about Python?}
  \begin{itemize}
    \item portable
    \item efficient
    \item extendible
  \end{itemize}
\end{frame}

\begin{frame}[fragile]
  \frametitle{Revision control in one slide}
  \begin{lstlisting}[basicstyle=\large]
> git init
(master #)> # begin development
(master #)> git add <files>
(master +)> git commit
(master)> # continue development
(master *)> git add <changed-files>
(master +)> git commit
\end{lstlisting}
  \scriptsize{Add the following to {\tt$\sim$/.bashrc}}
  \begin{lstlisting}[language=bash]
GIT_PS1_SHOWDIRTYSTATE=true
export PS1=[\u@\h \W(__git_ps1)]\$
\end{lstlisting}
\end{frame}

\begin{frame}[fragile]
  \frametitle{More advanced git commands}
  \begin{lstlisting}
(master)> git branch new-branch
(master)> git checkout new-branch
(new-branch)> # independent development continues
(new-branch)> git checkout master
(master)> git merge new-branch
(master)> # follow directions
\end{lstlisting}
Branches can be local or ``remote''
\end{frame}


\begin{frame}[fragile]
  \frametitle{Disclaimer}
  Python 2.7 $!=$ 3.0
  \vskip1cm
  \begin{verbatim}
http://wiki.python.org/moin/Python2orPython3
\end{verbatim}
\end{frame}

\begin{frame}[fragile]
  \frametitle{Read the documentation}
  \begin{verbatim}
> pydoc

> python
>>> help()
\end{verbatim}
\end{frame}

\begin{frame}[fragile]
  \frametitle{Python objects, types and values}
  \begin{itemize}
    \item Variables are just ``pointers'' to {\tt objects}
      \begin{itemize}
        \item {\it everything} is an {\tt object}
        \item Not all objects have methods or are subclassable
        \item All objects can be passed to functions.
        \item Every object has a unique ``id''
          \begin{itemize}
            \item The ``is'' operator determines if variables point to the same object
          \end{itemize}
      \end{itemize}
    \item Object types in python are mostly implicit
    \item The type of the {\tt object} to which a variable points can change during the program
  \end{itemize}
\end{frame}



\begin{frame}[fragile]
  \frametitle{Numerical types}
  \begin{itemize}
    \item Integers
      \begin{itemize}
        \item \lstinline[language=python]|a = 1|
      \end{itemize}
    \item Floats
      \begin{itemize}
        \item \lstinline[language=python]|a = 1.0|
        \item {\tt numpy} provides precision control
      \end{itemize}
    \item Complex numbers
      \begin{itemize}
        \item \lstinline[language=python]|a = 1.5 + 0.5j|
        \item \lstinline[language=python]|a.real| and \lstinline[language=python]|a.imag| return each component
      \end{itemize}
    \item Type casts
      \begin{itemize}
        \item \lstinline[language=python]|myFloat = float(myInteger)|
        \item \lstinline[language=python]|myInt = int(myFloat)|
      \end{itemize}
    \item Operators
      \begin{itemize}
        \item Addition \lstinline[language=python]|+|
        \item Subtraction \lstinline[language=python]|-|
        \item Multiplication \lstinline[language=python]|*|
        \item Exponentiation \lstinline[language=python]|**|
        \item Division \lstinline[language=python]|/|
        \item Modulus \lstinline[language=python]|%|
      \end{itemize}
  \end{itemize}
\end{frame}

\begin{frame}
  \frametitle{Other useful types}
  \begin{itemize}
    \item Boolean
      \begin{itemize}
        \item \lstinline[language=python]|a = (3 > 4)|
        \item \lstinline[language=python]|not|, \lstinline[language=python]|and|, \lstinline[language=python]|or|
      \end{itemize}
    \item Strings
      \begin{itemize}
        \item \lstinline[language=python]|a = 'single quoted'|
        \item \lstinline[language=python]|a = "double quoted"|
        \item \lstinline[language=python]|doubleA = float(a)|
        \item \lstinline[language=python]|intA = int(a)|
        \item \lstinline[language=python]|wordList = a.split()|
        \item ``\symbol{92}'' is the escape character
      \end{itemize}
  \end{itemize}
\end{frame}

\begin{frame}
  \frametitle{Containers (sequence)}
  All are indexed starting at 0 and are arguments for {\tt len()}
  \begin{itemize}
    \item one-dimensional list
      \begin{itemize}
        \item \lstinline[language=python]|List = ['red','green','blue']|
        \item \lstinline[language=python]|type(L[1])| returns {\tt <type 'str'>}
        \item Resizable with \lstinline[language=python]|L.append('ultraviolet')| or \lstinline[language=python]|L.insert(0,'infrared')|
        \item ``\lstinline[language=python]|+|'' will concatenate lists!
        \item Generally homogeneous data
        \item Lists are not arrays
      \end{itemize}
    \item Tuples
      \begin{itemize}
        \item \lstinline[language=python]|Tuple = (1,[3.,5.,7.],'red')|
        \item Generally inhomogeneous data
        \item There is no append or insert
      \end{itemize}
    \item Dictionaries
      \begin{itemize}
        \item Key - value pairs\\
          \lstinline[language=python]|Elements = \{'Oxygen': 8, 'Hydrogen': 1\}|\\
          \lstinline[language=python]|Elements['Carbon'] = 6|\\
          \begin{itemize}
            \item \lstinline[language=python]|Elements.keys()| returns element names and \lstinline[language=python]|Elements.values()| returns Z.
            \item Any type can be a key or value
          \end{itemize}
      \end{itemize}
  \end{itemize}
\end{frame}

\begin{frame}[fragile]
  \frametitle{Tuples or Lists?}
  \begin{itemize}
    \item Lists storing homogeneous elements allows expansion of the list
      \begin{itemize}
        \item The code does the order of elements
        \item Elements can be added or deleted
        \item Used for iteration
      \end{itemize}
      \vskip1cm
    \item Tuples storing inhomogeneous elements are like {\tt structs}
      \begin{itemize}
        \item Code written for tuples {\it requires} knowledge of the tuple
        \item Generally just a convenience container
      \end{itemize}
      \vskip1cm
    \item In somcases mutability will be important
  \end{itemize}
\end{frame}

\begin{frame}[fragile]
  \frametitle{Container operations}
  \begin{itemize}
    \item Slicing just like Fortran 90\\
      \lstinline[language=python]|L[start:stop:stride]|\\
      $\mathrm{start} \leq i < \mathrm{stop}; i+=\mathrm{stride}$
    \item Negative indices start at the end of the list
      \begin{itemize}
        \item -1 is the last element
      \end{itemize}
    \item Search for value with ``in''\\
      \lstinline[language=python]|print (5 in L)|
    \item Find the element with {\tt index}\\
      \lstinline[language=python]|'string'.index('n')|
  \end{itemize}
\end{frame}

\begin{frame}[fragile]
  \frametitle{Other operations}
  \begin{itemize}
    \item Search\\
      \begin{lstlisting}[language=python]
        >>> List = [1,'red',3.0,(4>3)]
        >>> print ('red' in List)
        True
      \end{lstlisting}
    \item Slice assignment\\
      \lstinline[language=python]|L[i:j] = i|
    \item Deletions\\
      \lstinline[language=python]|del L[i:j]|
  \end{itemize}

\end{frame}

\begin{frame}[fragile]
  \frametitle{Mutability of objects}
  \begin{itemize}
    \item Immutable objects get created and destroyed upon assignment and collection
      \begin{itemize}
        \item Strings
        \item Numbers
        \item Tuples
      \end{itemize}
    \item Mutabale objects create references to objects upon assignment
      \begin{itemize}
        \item Lists
        \item Dictionaries
      \end{itemize}
  \end{itemize}
\end{frame}

\begin{frame}[fragile]
  \frametitle{Loops}
  \begin{itemize}
    \item Iterate over any sequence
      \begin{itemize}
        \item string, list, keys in dictionary, lines in file\\
  \begin{lstlisting}[language=python]
vowels = 'aeiouy'
for i in 'orbital':
    if i in vowels:
        print(i)
\end{lstlisting}
\end{itemize}
  \item Keep the counter\\
\begin{lstlisting}[language=python]
shells = ('s', 'p', 'd', 'f')
for index, thisShell in enumerate(shells):
    print index, thisShell
    \end{lstlisting}
  \item List comprehension\\
\begin{lstlisting}[language=python]
L3 = [i**3 for i in range(4)]
\end{lstlisting}
\begin{itemize}
  \item {\tt range(start,stop,stride)}
    \begin{itemize}
      \item end point is omitted
    \end{itemize}
\end{itemize}
\end{itemize}
\end{frame}

\begin{frame}
  \frametitle{Modules}
  \begin{itemize}
    \item Import {\tt math.py} such that {\tt math} becomes the object name\\
      \lstinline[language=python]|import math|\\
      \lstinline[language=python]|print math.pi|\\
      \lstinline[language=python]|print math.sin(math.pi)|
    \item Alternatives
      \begin{itemize}
        \item \lstinline[language=python]|from math import sin|
        \item \lstinline[language=python]|import math as maths|
      \end{itemize}
    \item Avoid
      \begin{itemize}
        \item \lstinline[language=python]|from math import *|
      \end{itemize}
  \end{itemize}
  \vskip1cm
{\it If you can imagine it, someone probably has a module that can do it.}\\
{\scriptsize \url{http://docs.python.org/2/py-modindex.html}}\\
{\scriptsize \url{http://wiki.python.org/moin/UsefulModules}}\\
\end{frame}

\begin{frame}[fragile]
  \frametitle{Receiving Input}
  \begin{lstlisting}[language=python]
import sys

try:
  inputXYZ = open(sys.argv[1],'r')
except:
  print "Please specify a file name"
  sys.exit(1)

try:
  nAtoms = int(inputXYZ.readline())
except ValueError:
  print "The first line is not an integer"
  sys.exit(1)

comment = inputXYZ.readline()
print comment

Atoms = [tuple(thisAtom.split()[0:4]) for thisAtom in inputXYZ]
  \end{lstlisting}

\end{frame}

\begin{frame}
  \frametitle{Mathematical Exercise}
  Compute Pi using the Wallis formula
  \vskip1cm
  \begin{equation*}
    \pi = 2\prod^{\infty}_{i=1}\frac{4i^2}{4i^2-1}
  \end{equation*}
\end{frame}

\end{document}
