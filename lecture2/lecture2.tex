\documentclass[xcolor=table,10pt,final]{beamer}
\renewcommand\mathfamilydefault{\rmdefault}

\setbeamertemplate{navigation symbols}{}
\usepackage{amsmath,amsfonts,amssymb,pxfonts,xspace}
\usepackage{textcomp}
\usepackage{lmodern}
\usepackage{verbatim}
\usepackage{graphicx}
\usepackage{listings}
\usepackage[T1]{fontenc}

\lstset{
    basicstyle=\footnotesize,
    keywordstyle=\color[rgb]{0.1,0.8,0.1}\bfseries,
    commentstyle=\color{blue},
    numbers=left,
    stringstyle=\ttfamily\color{red!50!brown},
    showstringspaces=false}
\lstset{literate=%
   *{0}{{{\color{red!20!violet}0}}}1
    {1}{{{\color{red!20!violet}1}}}1
    {2}{{{\color{red!20!violet}2}}}1
    {3}{{{\color{red!20!violet}3}}}1
    {4}{{{\color{red!20!violet}4}}}1
    {5}{{{\color{red!20!violet}5}}}1
    {6}{{{\color{red!20!violet}6}}}1
    {7}{{{\color{red!20!violet}7}}}1
    {8}{{{\color{red!20!violet}8}}}1
    {9}{{{\color{red!20!violet}9}}}1
}



\begin{document}

\title{Python for Scientific Computing}
\subtitle{Lecture 2: Data Structures}
\author{Albert DeFusco\\Center for Simulation and Modeling}
\date{\today}
\frame{\titlepage}

\begin{frame}
  \frametitle{Download your own Python}
  \url{https://www.enthought.com/products/epd/free/}
\end{frame}

\begin{frame}[fragile]
  \frametitle{Review $\pi$}
  \begin{equation*}
    \pi = 2\prod^{\infty}_{i=1}\frac{4i^2}{4i^2-1}
  \end{equation*}
  \pause
\begin{lstlisting}[language=Python]
pi = 2.
for i in range(1,n):
  pi *= 4.*i**2 / (4.*i**2 - 1)
print pi
\end{lstlisting}
\end{frame}

\begin{frame}
  \frametitle{Containers}
  \begin{itemize}
    \item tuples, lists and strings
  \begin{itemize}
    \item Elements are accessed with {\tt container[ ]}
      \begin{itemize}
        \item Indexed starting at 0
      \end{itemize}
    \item Arguments for {\tt len(container)}
    \item Each type has special features
    %\item Iterated with {\tt iter(container)}
  \end{itemize}
  \end{itemize}
\end{frame}

\begin{frame}
  \frametitle{Tuples}
  \begin{itemize}
    \item \lstinline[language=Python]|Tuple = (1,3.,'red')|
    \item Contain {\it references} to other objects
    \item The structure cannot be changed
    \item A convenient way to pass multiple objects
      \begin{itemize}
        \item Packing and un-packing
      \end{itemize}
  \end{itemize}
\end{frame}

\begin{frame}
  \frametitle{Lists}
  \begin{itemize}
    \item \lstinline[language=Python]|List = ['red','green','blue']|
    %\item \lstinline[language=Python]|type(L[1])| returns {\tt <type 'str'>}
    \item Resizable %with \lstinline[language=Python]|L.append('ultraviolet')| or \lstinline[language=Python]|L.insert(0,'infrared')|
    %\item ``\lstinline[language=Python]|+|'' will concatenate lists!
    \item List-of-Lists is a multi-dimensional list
    \item Lists are not arrays
    \item {\tt range()} is a list
  \end{itemize}
\end{frame}

\begin{frame}
  \frametitle{Dictionaries}
      \begin{itemize}
        \item Key - value pairs\\
          \lstinline[language=Python]|Protons = \{'Oxygen': 8, 'Hydrogen': 1\}|\\
          \lstinline[language=Python]|Protons['Carbon'] = 6|\\
          \begin{itemize}
          %  \item \lstinline[language=Python]|Protons.keys()| returns element names and \lstinline[language=Python]|Protons.values()| returns nuclear charge.
            \item Any type can be a key or value
            \item Look-up tables
            \item Sorting and searching operations
          \end{itemize}
      \end{itemize}
\end{frame}

\begin{frame}[fragile]
  \frametitle{Indexing Lists and tuples}
  \begin{itemize}
    \item Slicing just like Fortran 90\\
      \lstinline[language=Python]|L[start:stop:stride]|\\
      $\mathrm{start} \leq i < \mathrm{stop}; i+=\mathrm{stride}$
    \item Negative indices start at the end of the list
      \begin{itemize}
        \item -1 is the last element
      \end{itemize}
  \end{itemize}
\end{frame}
    %\item Slice assignment for lists\\
    %  \lstinline[language=Python]|L[i:j] = i|
    %\item Deletions\\
    %  \lstinline[language=Python]|del L[i:j]|

\begin{frame}[fragile]
  \frametitle{Other operations}
  \begin{itemize}
    \item Search for value with {\tt in}
    \item Concatenate with {\tt +} or {\tt *}
    \item Count number of occurrences
      %\lstinline[language=Python]|print (5 in L)|
    %\item Find the element with {\tt index}\\
    %  \lstinline[language=Python]|'string'.index('n')|
  \end{itemize}
\end{frame}

\begin{frame}[fragile]
  \frametitle{Loops}
  \begin{itemize}
    \item Iterate over any sequence
      \begin{itemize}
        \item string, list, keys in dictionary, lines in file
      \end{itemize}
  \end{itemize}
  \begin{lstlisting}[language=Python]
vowels = 'aeiouy'
for i in 'orbital':
    if i in vowels:
        print(i)
\end{lstlisting}
\end{frame}

\begin{frame}[fragile]
  \frametitle{Loops}
  \begin{itemize}
    \item Keep a counter
  \end{itemize}
\begin{lstlisting}[language=Python]
shells = ('s', 'p', 'd', 'f')
for index, thisShell in enumerate(shells):
    print index, thisShell
    \end{lstlisting}
\end{frame}

\begin{frame}[fragile]
  \frametitle{Loops}
  \begin{itemize}
    \item List comprehension
  \end{itemize}
\begin{lstlisting}[language=Python]
even = [i for i in range(100) if i%2 == 0]

listX = [-1, 0, 1]
listY = [2, 4]
myTuple = [(x,y) for x in listX for y in listY]
\end{lstlisting}
\end{frame}




\begin{frame}[fragile]
  \frametitle{Mutability of objects}
  \begin{itemize}
    \item Immutable objects get created and destroyed upon assignment and collection
      \begin{itemize}
        \item Strings
        \item Numbers (no ++ operator)
        \item Tuples
      \end{itemize}
    \item Mutable objects create references to contained objects upon assignment
      \begin{itemize}
        \item Lists
        \item Dictionaries
      \end{itemize}
  \end{itemize}
\end{frame}

\begin{frame}
  \frametitle{Hands-on: Mutability}
\end{frame}

\begin{frame}[fragile]
  \frametitle{Tuples or Lists?}
  \begin{itemize}
    \item List: homogeneous data
      \begin{itemize}
        \item Elements can be added or deleted
        \item Elements can be in any order
        \item Mutable
      \end{itemize}
      \vskip1cm
    \item Tuples: heterogeneous data ({\tt structs})
      \begin{itemize}
        \item Constant size
        \item Order matters
        \item Immutable
      \end{itemize}
  \end{itemize}
\end{frame}


\begin{frame}
  \frametitle{Container examples}
  \begin{itemize}
    \item List
      \begin{itemize}
        \item Particles
        \item Lines in an input file
      \end{itemize}
    \item Tuple
      \begin{itemize}
        \item Position data
      \end{itemize}
    \item Dictionary
      \begin{itemize}
        \item Associated lists
        \item Look-up tables
        \item Histograms
        \item Networks
        \item Graphs
      \end{itemize}
  \end{itemize}
\end{frame}

\end{document}
